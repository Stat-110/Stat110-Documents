\documentclass[]{article}
\usepackage{lmodern}
\usepackage{amssymb,amsmath}
\usepackage{ifxetex,ifluatex}
\usepackage{fixltx2e} % provides \textsubscript
\ifnum 0\ifxetex 1\fi\ifluatex 1\fi=0 % if pdftex
  \usepackage[T1]{fontenc}
  \usepackage[utf8]{inputenc}
\else % if luatex or xelatex
  \ifxetex
    \usepackage{mathspec}
  \else
    \usepackage{fontspec}
  \fi
  \defaultfontfeatures{Ligatures=TeX,Scale=MatchLowercase}
\fi
% use upquote if available, for straight quotes in verbatim environments
\IfFileExists{upquote.sty}{\usepackage{upquote}}{}
% use microtype if available
\IfFileExists{microtype.sty}{%
\usepackage{microtype}
\UseMicrotypeSet[protrusion]{basicmath} % disable protrusion for tt fonts
}{}
\usepackage[margin=1in]{geometry}
\usepackage{hyperref}
\hypersetup{unicode=true,
            pdftitle={Table Quiz \#1 (5 pts)},
            pdfborder={0 0 0},
            breaklinks=true}
\urlstyle{same}  % don't use monospace font for urls
\usepackage{graphicx,grffile}
\makeatletter
\def\maxwidth{\ifdim\Gin@nat@width>\linewidth\linewidth\else\Gin@nat@width\fi}
\def\maxheight{\ifdim\Gin@nat@height>\textheight\textheight\else\Gin@nat@height\fi}
\makeatother
% Scale images if necessary, so that they will not overflow the page
% margins by default, and it is still possible to overwrite the defaults
% using explicit options in \includegraphics[width, height, ...]{}
\setkeys{Gin}{width=\maxwidth,height=\maxheight,keepaspectratio}
\IfFileExists{parskip.sty}{%
\usepackage{parskip}
}{% else
\setlength{\parindent}{0pt}
\setlength{\parskip}{6pt plus 2pt minus 1pt}
}
\setlength{\emergencystretch}{3em}  % prevent overfull lines
\providecommand{\tightlist}{%
  \setlength{\itemsep}{0pt}\setlength{\parskip}{0pt}}
\setcounter{secnumdepth}{0}
% Redefines (sub)paragraphs to behave more like sections
\ifx\paragraph\undefined\else
\let\oldparagraph\paragraph
\renewcommand{\paragraph}[1]{\oldparagraph{#1}\mbox{}}
\fi
\ifx\subparagraph\undefined\else
\let\oldsubparagraph\subparagraph
\renewcommand{\subparagraph}[1]{\oldsubparagraph{#1}\mbox{}}
\fi

%%% Use protect on footnotes to avoid problems with footnotes in titles
\let\rmarkdownfootnote\footnote%
\def\footnote{\protect\rmarkdownfootnote}

%%% Change title format to be more compact
\usepackage{titling}

% Create subtitle command for use in maketitle
\providecommand{\subtitle}[1]{
  \posttitle{
    \begin{center}\large#1\end{center}
    }
}

\setlength{\droptitle}{-2em}

  \title{Table Quiz \#1 (5 pts)}
    \pretitle{\vspace{\droptitle}\centering\huge}
  \posttitle{\par}
    \author{}
    \preauthor{}\postauthor{}
    \date{}
    \predate{}\postdate{}
  

\begin{document}
\maketitle

\subsection{First \& Last Names:}\label{first-last-names}

\begin{enumerate}
\def\labelenumi{\arabic{enumi}.}
\item
  Which of these \textbf{do not} apply to the word \texttt{statistics}?

  \begin{enumerate}
  \def\labelenumii{\alph{enumii}.}
  \item
    Statistics are numbers measured for some purpose.
  \item
    Statistics is a collection of procedures for collecting amd
    analyzing data.
  \item
    Statistics is a tool to help you make decisions when faced with
    uncertainty.
  \end{enumerate}
\end{enumerate}

\(\color{red}{\text{d. All of the above statements apply to the word statistics.}}\)

~

\begin{enumerate}
\def\labelenumi{\arabic{enumi}.}
\setcounter{enumi}{1}
\item
  A recent sample of 1,000 American adults found that 39\% support Joe
  Biden to be the Democratic Presidential nominee. Which of the
  following describes the population for this example?

  \begin{enumerate}
  \def\labelenumii{\alph{enumii}.}
  \item
    The 1,000 American adults who participated in the study.
  \item
    All American adults who support Joe Biden.
  \item
    The 29\% of American adults in the sample who support Joe Biden.
  \item
    All American adults.
  \end{enumerate}
\end{enumerate}

~

\begin{enumerate}
\def\labelenumi{\arabic{enumi}.}
\setcounter{enumi}{2}
\item
  Suppose you are conducting an experiment that involves assigning each
  of 100 participants to one of two treatments: Treatment A or Treatment
  B. Which of the following would \textbf{not} be considered to be a
  random assignment of participants to treatments?

  \begin{enumerate}
  \def\labelenumii{\alph{enumii}.}
  \item
    For each participant, flip a coin. If the coin lands heads up,
    assign them to group A. If the coin lands tails up, assign them to
    Group B.
  \item
    Put all 100 names in a hat and mix them up thoroughly. Draw 50 names
    from the hat and assign them to Group A. Everyone else is assigned
    to Group B.
  \item
    As the paticipants show up for the study, assign the first 50 of
    them to Group A, and the last 50 to Group B.
  \end{enumerate}
\end{enumerate}

~

\begin{enumerate}
\def\labelenumi{\arabic{enumi}.}
\setcounter{enumi}{3}
\item
  Suppose you want to conduct a survey to determine who is most likely
  to win the next presidential election. Which of the following would be
  considered to be a representative (unbiased) sample?

  \begin{enumerate}
  \def\labelenumii{\alph{enumii}.}
  \item
    1,000 likely voters who called in to a local radio talk show.
  \item
    1,000 likely voters who returned surveys sent to everyone on a
    Democrat or Republican newsletter mailing list.
  \item
    1,000 likely voters who replied to an Internet website survey.
  \item
    None of these samples are representative.
  \end{enumerate}
\end{enumerate}

~

\begin{enumerate}
\def\labelenumi{\arabic{enumi}.}
\setcounter{enumi}{4}
\item
  Suppose a group of students who reported smoking marijuana was found
  to have significantly lower test scores than a group of students who
  reported they don't smoke marijuana. We can conclude that smoking
  marijuana leads to lower test scores.

  \begin{enumerate}
  \def\labelenumii{\alph{enumii}.}
  \item
    True
  \item
    False
  \end{enumerate}
\end{enumerate}


\end{document}
